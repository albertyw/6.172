\title{6.172 Project 2.2 Writeup}
\author{
    Albert Wang, Kaichen Ma
}
\date{\today}

\documentclass[12pt]{article}
\linespread{2}


\begin{document}
\maketitle
\section{4}


\subsection{4.1}
Original Output
\begin{verbatim}
Number of frames = 4000
1347 Line-Wall Collisions
20276 Line-Line Collisions
88.884 seconds in total
\end{verbatim}


\subsection{4.2}
This is not a problem for quadtree because the Line will just be stored by the 
inner quadtree node, not be a quadtree leaf.  This, however, means that the 
Line needs to be checked for collisions against each of the node's children 
(although this probably can be optimized to being just checked against the 
children that the large Line overlaps).  

This method of storing some line segments in inner nodes is not as 
efficient as the original quadtree implementation because inner node Lines need 
to be checked for collisions against children lines.  However, it is still more 
efficient than the original implementation because it still reduces the number of 
collision checks.  


\subsection{4.3}
Quadtree output
\begin{verbatim}
Number of frames = 4000
1297 Line-Wall Collisions
20094 Line-Line Collisions
18.238 seconds in total
\end{verbatim}
This speedup is 4.9 times faster.  Since quadtree is supposed to dramatically 
reduce the number of collision checks, this speedup is expected.  


\subsection{4.4}
The quadtree function was implemented as a recursive function.  Each run of the 
function represents a single node of the quadtree.  Instead of using the 
CollisionWorld variable "lines", the quadtree function passes a vector of lines 
that it creates to children nodes.  When the quadtree function is first run, 
the quadtree sorts every line that it gets into one of five vectors: a leaf 
vector if the line can't be stored in a child of the node, or one of the vectors 
that is sent to one of the node's quadrant children.  The quadtree function then 
checks each of the four quadrant vectors to make sure they have more than 1 line, 
then instantiates the quadtree function for the child node.  Lastly, the quadtree 
function runs collision checks between Lines in its own vector and Lines in its 
children.  

Each quadtree node has one 
vector<Line*> variable that stores all the lines.  Quadtree nodes will create 
the vectors for their children nodes.  This might be optimized to use an object 
that is not a vector.


\subsection{4.5}
Max elements without subdividing.

Found that the best time to stop subdividing quadtrees into children nodes 
was when the number of Lines in a quadtree was 10.  
\begin{verbatim}
Number of frames = 4000
1310 Line-Wall Collisions
20302 Line-Line Collisions
17.399 seconds in total
\end{verbatim}
This is a 5\% speedup.  


\subsection{4.6}
Max recursion depth

Maximum recursion depth doesn't seem to matter because any benefits you get 
from maximum recursion depth are already gained by implementing a maximum ele


\subsection{4.7}
One part that was optimized was that there are now two functions that manually 
compare vectors of lines.  Since there are some parts of the code that need to 
compare two different vectors and some parts that compare vectors to themselves, 
the parts that compare vectors against themselves can be optimized to not 
recompare Lines that have already been recompared.  This can probably be 
optimized for the former comparisons too so that lines are never compared for 
collisions more than once in a timestep.  




\section{5}
\subsection{5.1}
Top six longest functions:
  24.55      8.89     8.89 1919921812     0.00     0.00  direction(Vec, Vec, Vec)
  23.55     17.41     8.53 1919921812     0.00     0.00  crossProduct(double, double, double, double)
  17.20     23.64     6.23 383984338     0.00     0.00  intersectLines(Vec, Vec, Vec, Vec)
   6.66     26.05     2.41 95996098     0.00     0.00  intersect(Line*, Line*, double)
   3.73     27.40     1.35 95996115     0.00     0.00  pointInParallelogram(Vec, Vec, Vec, Vec, Vec)
   3.15     28.54     1.14 287994792     0.00     0.00  Vec::operator-(Vec)


\subsection{5.2}
Changed detectIntersectionNew and detectIntersectionNewSame functions to save 
collisions to a list instead of calling collisionSolver directly.  

New time:
\begin{verbatim}
Number of frames = 4000
1265 Line-Wall Collisions
20250 Line-Line Collisions
17.792 seconds in total
\end{verbatim}


\subsection{5.3}
Made the tester into a cilk\_for and changed the list into a reducer.  

New time:
\begin{verbatim}
Number of frames = 4000
1265 Line-Wall Collisions
20250 Line-Line Collisions
10.734 seconds in total
\end{verbatim}


\subsection{5.4}
Optimized some parts of code to run faster.  
Added cilk\_spawn to the functions that start the search for intersections.  
\begin{verbatim}
Number of frames = 4000
1227 Line-Wall Collisions
18999 Line-Line Collisions
8.978 seconds in total
\end{verbatim}

cilkscreen says no races.  Runs are the same time.  
\subsection{5.5}
Span is 337 million instructions

Work is 17986 million insturctions

Parallelism is 53.34

Best maximum recursion depth is 7.  Best size of quadtrees is 25 or less.  Speed
was:
\begin{verbatim}
Number of frames = 4000
1267 Line-Wall Collisions
20383 Line-Line Collisions
6.846 seconds in total
\end{verbatim}


\subsection{5.6}
Total speedup from beginning was about 10 times faster from original code.  


We want to basically parallelize calls to exterior functions, especially to 
collisionsolver() and intersect().  We may also look into modifying the intersect 
function itself to run faster.  We currently have problems with iterators because 
cilk\_for cannot use STL list iterators while the output of a reducer\_list is a 
list, and not a vector.  Unless there is a way to easily convert from STL lists 
to vectors, we may not be able to parallelize some loops.  

Albert wrote the quadtree, detectIntersectionNew, and allCollisionSolver functions.  
Kaichen did testing and wrote the lineInsideQuadrant function.  

\end{document}
