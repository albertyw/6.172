\title{6.172 Project 1 Final Writeup}
\author{
    Albert Wang, Lekha Kuhananthan
}
\date{\today}

\documentclass[12pt]{article}
\linespread{2}


\begin{document}
\maketitle

The bitarray code has changed dramatically since the beta submission.  For the 
beta, bitarray accessed bits individually and moved them to their proper spots.  
Although this was an improvement over the original implementation, the code ran
in 1.350 ms.  For the final, instead of moving bits individually, the code moved bytes
whenever possible. After testing and optimization, the final code runs in 0.179 ms, a 7.5 times improvement.  

After the final was first finished and correct, the code ran in 0.279 ms, an 
only 5 times improvement.  From there, we inlined the shift\_byte function, which
is called multiple times.  This inlining dropped the time down to 0.270 ms. 
Computation of constants in a for-loop were removed to drop the time to 0.268ms.
Three for-loops were optimized to increment a pointer to the bitarray instead of
incrementing the a bitarray index variable, which dropped the time to 0.191 ms.  Two loops were combined, which 
resulted in a running time of 0.181 ms. Finally, the the code was cleaned up to remove
unused functions and statements, which finished with a running time of 0.179 ms.  

Most of these code improvements came from various class and recitation ideas.  
Switching to modifying mostly on the byte-level instead of the bit-level was actually 
thought of right before the beta was due, but there wasn't any time to implement 
the new strategy then.

Master Annirudda Bohra gave some good advice about the project.  Although there 
weren't any explicit tips about speeding up the beta code, there were some 
more practical tips such as coding style and inlining.  Some of the more useful 
tips included that the code should stay away from multiplication as much as possible, 
and to experiment with changing the code. 

Master Dan Speicher spent much time emphasizing the importance of code readability. 
He explained that, especially in group projects, comments should
do more than just outline code - bit hacks should be explained so that those unaware
of the bit hack do not need to waste time figuring it out. In addition, comments
are important when revisiting code. When rereading code several months in the future, it is highly
unlikely that even the code's author will remember the meaning of all that code.

\end{document}
