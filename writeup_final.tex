\title{6.172 Project 1 Final Writeup}
\author{
    Albert Wang, Lekha Kuhananthan
}
\date{\today}

\documentclass[12pt]{article}
\linespread{2}


\begin{document}
\maketitle

The bitarray code has changed dramatically since the beta submission.  For the 
beta, bitarray accessed bits individually and moved them to their proper spots.  
Although this was an improvement over the original implementation, the code ran
in 1.350 ms.  For the final, instead of moving bits individually, the code tried 
to move as much data through byte accesses.  When the final code was tested and 
optimized, 
the final code finished in 0.179 ms, a 7.5 times improvement.  

After the final was first finished and correct, the code ran in 0.279 ms, an 
only 5 times improvement.  From there, we inlined a shift\_byte function, which is 
called many times in our code.  This inlining dropped the time down to 0.270 ms.  
Three loops were also optimized to increment a pointer directly, instead of an extra
index variable, which dropped the time to 0.210 ms.  A function call was also 
removed to drop the running time to 0.191 ms.  Two loops were combined, which 
resulted in a running time of 0.181 ms, then the code was cleaned up to remove
unused functions and statements which finished with a running time of 0.179 ms.  

Most of these code improvements came from various class and recitation ideas.  
Switching to modifying mostly on the byte-level instead of the bit-level was actually 
thought of right before the beta was due, but there wasn't any time to implement 
the new strategy.  

Master Annirudda Bohra gave some good advice about the project.  Although there 
weren't any explicit tips about speeding up the beta code, there were some 
more practical tips such as coding style and inlining.  Some of the more useful 
tips included that the code should stay away from multiplication as much as possible, 
and to experiment with changing the code.  

\end{document}
